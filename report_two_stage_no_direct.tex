\documentclass[journal,onecolumn]{IEEEtran}
\usepackage{amsmath,amsfonts,amssymb,bm}
\usepackage{geometry}
\geometry{margin=1in}
\title{Two-Stage Scattering Matrix Design with Direct Link}
\author{Hesam Nezhadmohammad}
\date{}

\begin{document}
\maketitle

\section{System Model in the Original Paper}

Consider the BD-RIS aided MU-MISO system in \cite{Fang24}, where a BS with $L$ antennas serves $K$ single-antenna users through a BD-RIS with $N$ elements. The paper models three channels:
\begin{itemize}
    \item the direct BS--user channel $\mathbf{G} \in \mathbb{C}^{L \times K}$,
    \item the BS--RIS channel $\mathbf{E} \in \mathbb{C}^{N \times L}$,
    \item the RIS--user channel $\mathbf{H} \in \mathbb{C}^{N \times K}$.
\end{itemize}
The BD-RIS is described by an $N \times N$ scattering matrix $\bm{\Theta}$ that is required, for the fully connected architecture, to satisfy the symmetric unitary constraints
\begin{equation}
    \mathbf{\Theta} = \mathbf{\Theta}^T,\,\mathbf{\Theta}^H \mathbf{\Theta} = \mathbf{I}_{N}.
\end{equation}

Let the effective channel be
\begin{equation}
    \mathbf{F}(\bm{\Theta}) = \mathbf{G}^H + \mathbf{H}^H \bm{\Theta} \mathbf{E}
    \label{eq:effective-channel}
\end{equation}
which is exactly the sum of the direct link and the RIS-reflected link. To obtain a passive beamforming matrix that is good for the later active beamforming step, \cite{Fang24} relaxes the original nonconvex constrained problem to the following Frobenius-norm maximization:

\begin{subequations}
    \label{eq:relaxed-problem}
    \begin{align}
        \max_{\mathbf{\Theta}}~~& f(\mathbf{\Theta}) \triangleq \left\| \mathbf{G}^H + \mathbf{H}^H \bm{\Theta} \mathbf{E} \right\|_F^2\\
        \mathrm{s.t.}~~~&\|\mathbf{\Theta}\|_F^2 \le N.
    \end{align}
\end{subequations}

% \begin{equation}
%     \max_{\bm{\Theta}} \; f(\bm{\Theta}) \triangleq \left\| \mathbf{G}^H + \mathbf{H}^H \bm{\Theta} \mathbf{E} \right\|_F^2
%     \quad \text{s.t.} \quad \|\bm{\Theta}\|_F^2 \le N.
%     \label{eq:relaxed-problem}
% \end{equation}
This is equation (3) in the paper, written here directly in matrix form. The paper then vectorizes it and reaches the same form. The key observation is that $f(\bm{\Theta})$ is smooth and quadratic in $\bm{\Theta}$.

\section{Gradient in the original model}

Define
\begin{equation}
    \mathbf{A}(\bm{\Theta}) \triangleq \mathbf{G}^H + \mathbf{H}^H \bm{\Theta} \mathbf{E}.
\end{equation}
Then
\begin{equation}
    f(\bm{\Theta}) = \|\mathbf{A}(\bm{\Theta})\|_F^2 = \operatorname{Tr}\left( \mathbf{A}(\bm{\Theta}) \mathbf{A}(\bm{\Theta})^H \right).
\end{equation}
Because $\mathbf{A}(\bm{\Theta})$ is affine in $\bm{\Theta}$, differentiating at $\bm{\Theta} = \mathbf{0}$ is simple. Let us write explicitly
\begin{equation}
    \mathbf{A}(\mathbf{0}) = \mathbf{G}^H,
    \qquad
    \left. \frac{\partial \mathbf{A}}{\partial \bm{\Theta}} \right|_{\bm{\Theta}=\mathbf{0}} : \Delta \bm{\Theta} \mapsto \mathbf{H}^H \Delta \bm{\Theta} \mathbf{E}.
\end{equation}
Using standard matrix calculus for a function of the form $\|\mathbf{G}^H + \mathbf{H}^H \bm{\Theta} \mathbf{E}\|_F^2$, the gradient at $\bm{\Theta}=\mathbf{0}$ is
\begin{equation}
    \nabla_{\bm{\Theta}} f(\mathbf{0}) = \mathbf{H} \mathbf{G}^H \mathbf{E}^H,
    \label{eq:original-gradient}
\end{equation}
which is exactly the expression the paper gives as equation (6) \cite{Fang24}.

The paper then chooses the direction
\begin{equation}
    \mathbf{D}_0 = \nabla_{\bm{\Theta}} f(\mathbf{0}) = \mathbf{H} \mathbf{G}^H \mathbf{E}^H
\end{equation}
and rescales it to satisfy the sphere constraint $\|\bm{\Theta}\|_F^2 \le N$. Finally, because the actual feasible set for a fully connected BD-RIS is the symmetric unitary set, the paper projects this matrix onto that set by the symmetric-unitary projection:
\begin{equation}
    \bm{\Theta} = \operatorname{symuni}\bigl( \mathbf{H} \mathbf{G}^H \mathbf{E}^H \bigr).
    \label{eq:original-symuni}
\end{equation}
This is the form reported as equation (15) in the paper \cite{Fang24}.

\section{Modified system: no direct BS--user channel}

Now suppose the direct channel is no longer present. That is, the term $\mathbf{G}^H$ in \eqref{eq:effective_channel} is removed and the users are served solely through the cascaded BS--RIS--user path. The effective channel reduces to
\begin{equation}
    \mathbf{F}_{\text{no-dir}}(\bm{\Theta}) = \mathbf{H}^H \bm{\Theta} \mathbf{E}.
    \label{eq:effective-no-direct}
\end{equation}
A natural analogue of the relaxed problem \eqref{eq:relaxed-problem} is then
\begin{subequations}
    \begin{align}
        \max_{\mathbf{\Theta}} ~~ &f_{\text{no-dir}}(\mathbf{\Theta}) \triangleq \left\| \mathbf{H}^H \mathbf{\Theta} \mathbf{E} \right\|_F^2\\
        \text{s.t.} ~~~& \|\mathbf{\Theta}\|_F^2 \le N.
        \label{eq:relaxed-no-direct}
    \end{align}
\end{subequations}

\subsection{Why the old gradient cannot be used}

If we differentiate \eqref{eq:relaxed-no-direct} at the same point $\bm{\Theta}_0 = \mathbf{0}$, we get
\begin{equation}
    \mathbf{F}_{\text{no-dir}}(\mathbf{0}) = \mathbf{0}
    \quad\Rightarrow\quad
    f_{\text{no-dir}}(\mathbf{0}) = 0.
\end{equation}
Let us compute its gradient. Define
\begin{equation}
    \mathbf{B}(\bm{\Theta}) \triangleq \mathbf{H}^H \bm{\Theta} \mathbf{E},
\end{equation}
so that
\begin{equation}
    f_{\text{no-dir}}(\bm{\Theta}) = \operatorname{Tr}\bigl( \mathbf{B}(\bm{\Theta}) \mathbf{B}(\bm{\Theta})^H \bigr).
\end{equation}
The differential is
\begin{equation}
    \mathrm{d} f_{\text{no-dir}} = 2 \operatorname{Re} \left\{ \operatorname{Tr}\bigl( \mathbf{B}(\bm{\Theta})^H \, \mathbf{H}^H (\mathrm{d}\bm{\Theta}) \mathbf{E} \bigr) \right\}.
\end{equation}
At $\bm{\Theta} = \mathbf{0}$ we have $\mathbf{B}(\mathbf{0}) = \mathbf{0}$, so the whole differential vanishes:
\begin{equation}
    \left. \nabla_{\bm{\Theta}} f_{\text{no-dir}}(\bm{\Theta}) \right|_{\bm{\Theta}=\mathbf{0}} = \mathbf{0}_{N \times N}.
    \label{eq:zero-gradient}
\end{equation}
This is the key difference from the original model. Once we remove $\mathbf{G}$, the origin becomes a flat point and no longer yields a usable direction.

\subsection{Choosing a feasible initial point}

A simple way to recover a meaningful direction is to evaluate the gradient at a \emph{feasible} point, not at the origin. A natural feasible point for the fully connected BD-RIS is
\begin{equation}
    \bm{\Theta}_0 = \mathbf{I}_N,
\end{equation}
which is symmetric and unitary. Evaluate $f_{\text{no-dir}}$ at $\bm{\Theta}_0$:
\begin{equation}
    f_{\text{no-dir}}(\mathbf{I}_N) = \|\mathbf{H}^H \mathbf{E}\|_F^2.
\end{equation}
Now differentiate $f_{\text{no-dir}}(\bm{\Theta})$ at $\bm{\Theta} = \mathbf{I}_N$. Using
\begin{equation}
    \mathbf{B}(\bm{\Theta}) = \mathbf{H}^H \bm{\Theta} \mathbf{E},
    \qquad
    \mathrm{d} \mathbf{B} = \mathbf{H}^H (\mathrm{d}\bm{\Theta}) \mathbf{E},
\end{equation}
we have
\begin{align}
    \mathrm{d} f_{\text{no-dir}}
    &= 2 \operatorname{Re} \left\{ \operatorname{Tr}\bigl( \mathbf{B}(\bm{\Theta})^H \, \mathbf{H}^H (\mathrm{d}\bm{\Theta}) \mathbf{E} \bigr) \right\} \\
    &= 2 \operatorname{Re} \left\{ \operatorname{Tr}\bigl( \mathbf{E}^H \bm{\Theta}^H \mathbf{H} \, \mathbf{H}^H (\mathrm{d}\bm{\Theta}) \mathbf{E} \bigr) \right\}.
\end{align}
At $\bm{\Theta} = \mathbf{I}_N$, this becomes
\begin{equation}
    \left. \mathrm{d} f_{\text{no-dir}} \right|_{\bm{\Theta}=\mathbf{I}_N}
    = 2 \operatorname{Re} \left\{ \operatorname{Tr}\bigl( \mathbf{E}^H \mathbf{H} \, \mathbf{H}^H (\mathrm{d}\bm{\Theta}) \mathbf{E} \bigr) \right\}.
\end{equation}
By cyclic property of the trace,
\begin{equation}
    \operatorname{Tr}\bigl( \mathbf{E}^H \mathbf{H} \, \mathbf{H}^H (\mathrm{d}\bm{\Theta}) \mathbf{E} \bigr)
    = \operatorname{Tr}\bigl( \mathbf{H} \mathbf{H}^H \mathbf{E} \mathbf{E}^H (\mathrm{d}\bm{\Theta}) \bigr).
\end{equation}
Therefore the gradient at $\bm{\Theta}=\mathbf{I}_N$ is
\begin{equation}
    \left. \nabla_{\bm{\Theta}} f_{\text{no-dir}}(\bm{\Theta}) \right|_{\bm{\Theta}=\mathbf{I}_N}
    = 2 \, \mathbf{H} \mathbf{H}^H \mathbf{E} \mathbf{E}^H.
    \label{eq:gradient-no-direct}
\end{equation}
The scalar factor 2 is irrelevant for the projection step (the paper itself notes that scaling does not change the symmetric-unitary projection \cite{Fang24}. Hence the direction is
\begin{equation}
    \mathbf{Z} \triangleq \mathbf{H} \mathbf{H}^H \mathbf{E} \mathbf{E}^H.
    \label{eq:Z-no-direct}
\end{equation}

\subsection{Low-complexity solution without the direct link}

The paper’s logic for the original model was:
\begin{enumerate}
    \item form a matrix that is proportional to the gradient at the chosen point;
    \item scale it to lie on the sphere $\|\bm{\Theta}\|_F^2 = N$;
    \item project it onto the symmetric unitary set by $\operatorname{symuni}(\cdot)$;
    \item use the resulting $\bm{\Theta}$ in the second stage.
\end{enumerate}
We can carry out the same logic here. Since $\operatorname{symuni}(\cdot)$ is homogeneous (i.e. $\operatorname{symuni}(\rho \mathbf{Z}) = \operatorname{symuni}(\mathbf{Z})$ for any $\rho \neq 0$, see Lemma 1 in the paper \cite{Fang24}), we can skip the explicit scaling and directly write the low-complexity passive beamforming matrix as
\begin{equation}
    \bm{\Theta}_{\text{no-dir}} = \operatorname{symuni}\bigl( \mathbf{H} \mathbf{H}^H \mathbf{E} \mathbf{E}^H \bigr).
    \label{eq:theta-no-direct}
\end{equation}
This is the exact analogue of equation (15) in \cite{Fang24}, with
\begin{equation}
    \mathbf{H} \mathbf{G}^H \mathbf{E}^H
    \quad \text{replaced by} \quad
    \mathbf{H} \mathbf{H}^H \mathbf{E} \mathbf{E}^H.
\end{equation}


\section{Group-connected case}

The paper extends the fully connected solution to the group-connected case by first forcing the matrix to be block diagonal with $N_g \times N_g$ blocks, and then applying $\operatorname{symuni}(\cdot)$ on each block, see its equations (16)--(17) \cite{Fang24}. If we denote by
\begin{equation}
    \operatorname{blkdiag}_{N_g}(\mathbf{Z})
\end{equation}
the operation that zeroes all off-group entries of $\mathbf{Z}$ (i.e. the Hadamard product with a block mask), then the group-connected analogue of \eqref{eq:theta-no-direct} is
\begin{equation}
    \bm{\Theta}_{\text{no-dir}}^{\text{(group)}}
    = \operatorname{diag} \bigl(
        \operatorname{symuni}(\mathbf{Z}_1),
        \dots,
        \operatorname{symuni}(\mathbf{Z}_G)
      \bigr),
    \label{eq:theta-no-direct-group}
\end{equation}
where
\begin{equation}
    \operatorname{blkdiag}_{N_g} \bigl( \mathbf{H} \mathbf{H}^H \mathbf{E} \mathbf{E}^H \bigr)
    = \operatorname{diag}(\mathbf{Z}_1, \dots, \mathbf{Z}_G),
\end{equation}
$G = N / N_g$, and each $\mathbf{Z}_g \in \mathbb{C}^{N_g \times N_g}$. This is exactly what MATLAB function
\texttt{group\_symuni} does: it masks the matrix to a block-diagonal one and then calls \texttt{symuni} on the result.

\section{Discussion}

The essential mathematical difference between the original paper’s setting and the ``no-direct-link’’ setting is this:
\begin{itemize}
    \item With $\mathbf{G} \neq \mathbf{0}$, the objective has a nonzero linear term at $\bm{\Theta}=\mathbf{0}$, which immediately produces the rank-$N$ matrix $\mathbf{H} \mathbf{G}^H \mathbf{E}^H$ as a ``good'' direction.
    \item With $\mathbf{G} = \mathbf{0}$, the objective is purely quadratic in $\bm{\Theta}$ and the origin becomes flat. To preserve the same low-complexity flavour, we move to a feasible point (the identity) and take the gradient there, which gives a matrix with the same structure but with channel correlation terms only.
\end{itemize}
Because the symmetric-unitary projection is a nearest-point projection in Frobenius norm onto the feasible BD-RIS set, using \eqref{eq:theta-no-direct} is consistent with the logic of \cite{Fang24}: we take the dominant information embedded in the channels and project it to the physically realisable scattering matrix.



\bibliographystyle{IEEEtran}
\bibliography{IEEEabrv,refs}


\end{document}
